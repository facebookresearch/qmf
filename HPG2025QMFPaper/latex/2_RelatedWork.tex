\vspace{-1.5mm}
\section{Related Work}

Unsupervised machine learning methods such as Lasso \cite{tibshirani1996regression}, non-negative matrix factorization \cite{lee1999learning,hoyer2004non} and overcomplete dictionary learning \cite{aharon2006} discover sparse structures in input data. The theoretical and algorithmic aspects of sparse matrix factorization have been also studied in depth \cite{bach2008convex,neyshabur2013sparse,gribonval2010dictionary}. Quantization of low-rank matrix approximation has also been recently considered in the context of compression \cite{saha2023matrixcompressionrandomizedlow}. Our specific case of blendshapes, however, presents particular geometric challenges, since high visual quality of the resulting facial animations is one of our main goals.

In computer animation, the traditional Facial Action Coding System (FACS) \cite{ekman1978facial} and its variants have long been the foundation of facial blendshape models. However, these systems are limited by their complex controls and requirement for a large number of blendshapes to achieve high fidelity \cite{kim2021optimizing,choi2022animatomy}. 
In cases where training data or facial rigs are available, classical statistical methods \cite{meyer2007key} and deep learning approaches \cite{bailey2020fast,chandran2022facial,dafei2023neural} can be employed to derive more efficient models. Our approach does not require any neural networks, even though our quantized matrix factorization utilizes the same underlying techniques such as the popular Adam optimizer \cite{kingma2014adam}.

Blendshape compression is also related to the problem of skinning decomposition, first introduced by \citet{james2005skinning} in the context of linear blend skinning. Subsequent research led to improvements in approximating the global optimum, starting with the addition of bone rigidity constraints by \cite{kavan2007skinning} or fitting input animations with hierarchical skeletons, as demonstrated by \citet{schaefer2007example}. Later, \citet{kavan2010fast} proposed an alternating solver achieving better approximations of the unknown global optimum, while \citet{hasler2010learning,bharaj2012automatically} further advanced the utilization of hierarchical skeletons. The benefits of adding bone rigidity constraints were further studied by \citet{le2012smooth}, which formed the basis for the popular Dem Bones implementation \cite{dembones}. Beyond classical skinning methods, \citet{seo2011compression} investigated blendshape compression via hierarchical semi-separable matrices and \citet{le2013two} proposed two-layer skinning for compressing skinning weights. High-quality jaw opening can be achieved without skeletons using "intermediate" blendshapes, as shown by \citet{lewis2014practice}, or with hierarchical skeletons, as demonstrated by \citet{tianye2017flame}. Finally, GPU-specific methods have been proposed to speed up blendshape evaluation using compute shaders \cite{costigan2016improving}, and deep learning-based approaches have been explored for fitting hierarchical skeletons \cite{moutafidou2023deep}.  Our primary point of comparison for the paper will be a recent state-of-the-art skinning decomposition-based compression method: Compressed Skinning for Facial Blendshapes (\csfb) \cite{kavan2024compressed}. Our proposed matrix factorization trades the traditional skinning formulation for higher performance.

Blendshapes are not the only way to animate faces; recent advances in anatomically-inspired systems and implicit neural representations have shown great promise in modeling complex geometries, including the human face. Notably \citet{choi2022animatomy} presented Animatomy, a novel representation of the human face that uses muscle fiber curves as an anatomical basis for fine-grained parameterization of facial expressions. \citet{Chandran_2024_CVPR} proposed Implicit Face Models, which learn to jointly model the facial anatomy and skin surface using an ensemble of implicit neural networks. However, blendshape models are orders of magnitude computationally less expensive, and so real-time systems on resource-constrained platforms continue to rely on blendshapes.


