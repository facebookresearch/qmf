\documentclass[acmtog,anonymous,review]{acmart}
\acmSubmissionID{papers\_1532}

\usepackage{booktabs} % For formal tables
\usepackage{wrapfig}
\usepackage{listings}
\usepackage{tikz}
\usepackage{nicematrix} % to illustrate real-time
\usepackage{graphicx}
\usepackage{subcaption}
\usepackage{hyperref}
\usepackage{xspace}
% \usepackage{cite}
% \usepackage{amsmath}
% \usepackage{amsfonts}
\usepackage{siunitx}
%\usepackage{hyperref}

\graphicspath{ {./images/} {./generated/} }

\lstset{language=C++,
                basicstyle=\ttfamily,
                keywordstyle=\color{blue}\ttfamily,
                stringstyle=\color{red}\ttfamily,
                commentstyle=\color{green}\ttfamily,
                morecomment=[l][\color{magenta}]{\#}
}
\newsavebox{\codebox}
% TOG prefers author-name bib system with square brackets
\citestyle{acmauthoryear}
%\setcitestyle{nosort,square} % nosort to allow for manual chronological ordering



\usepackage[ruled]{algorithm2e} % For algorithms
\renewcommand{\algorithmcfname}{ALGORITHM}
\setlength{\algomargin}{0pt}
\SetAlFnt{\small}
\SetAlCapFnt{\small}
%\SetAlCapNameFnt{\small}https://overleaf.thefacebook.com/project/6778e7795bc9a40e2bfdb80e
\SetAlCapHSkip{0pt}

% Metadata Information
%\acmJournal{TOG}https://overleaf.thefacebook.com/project/6778e7795bc9a40e2bfdb80e


% 
\usepackage{xspace}
\makeatletter
\DeclareRobustCommand\onedot{\futurelet\@let@token\@onedot}
\def\@onedot{\ifx\@let@token.\else.\null\fi\xspace}
\def\eg{\emph{e.g}\onedot} \def\Eg{\emph{E.g}\onedot}
\def\ie{\emph{i.e}\onedot} \def\Ie{\emph{I.e}\onedot}
\def\cf{\emph{c.f}\onedot} \def\Cf{\emph{C.f}\onedot}
\def\etc{\emph{etc}\onedot} \def\vs{\emph{vs}\onedot}
\def\wrt{w.r.t\onedot} \def\dof{d.o.f\onedot}
%\newcommand{\etal}{\emph{et al}\onedot}
\def\etal{\emph{et al}\onedot}
\makeatother

%\usepackage[makeroom]{cancel}

\newcommand{\tderiv}[2]{\frac{d #1}{d #2}}
\newcommand{\ttderiv}[2]{\frac{d^2 #1}{d #2^2}}
\newcommand{\ttderivmixed}[3]{\frac{d^2 #1}{d #2 d #3}}
\newcommand{\pderiv}[2]{\frac{\partial #1}{\partial #2}}
\newcommand{\ppderiv}[2]{\frac{\partial^2 #1}{\partial #2^2}}
\newcommand{\ppderivmixed}[3]{\frac{\partial^2 #1}{\partial #2 \partial #3}}

% references and abbreviations
\newcommand{\refsec}[1]{Section~\ref{#1}}
\newcommand{\Section}[1]{Section~\ref{#1}}
\newcommand{\Sec}[1]{Sec.~\ref{#1}}
\newcommand{\Eq}[1]{Eq.~(\ref{#1})}
\newcommand{\Equation}[1]{Equation ~(\ref{#1})}
\newcommand{\Fig}[1]{Fig.~\ref{#1}}
\newcommand{\Figure}[1]{Figure~\ref{#1}}
\newcommand{\Subfigure}[2]{Figure~\ref{#1}(\subref{#2})}
\newcommand{\Subfig}[2]{Fig.~\ref{#1}(\subref{#2})}
\newcommand{\Table}[1]{Table~\ref{#1}}
\newcommand{\Appendix}[1]{Appendix~\ref{#1}}
\newcommand{\Alg}[1]{Alg.~\ref{#1}}
\newcommand{\Algorithm}[1]{Algorithm~\ref{#1}}


\newcommand{\A}{{\mathsf{A}}}
\newcommand{\dt}{{\Delta t}}
\renewcommand{\P}{{\mathsf{P}}}
\renewcommand{\u}{{\mathbf u}}
\newcommand{\f}{{\mathbf f}}
\newcommand{\bmu}{\boldsymbol{\mu}}
\newcommand{\bSigma}{\boldsymbol{\Sigma}}
\newcommand{\btheta}{\boldsymbol{\theta}}

% bold
\newcommand{\bA}{\mathbf{A}}
\newcommand{\bB}{\mathbf{B}}
\newcommand{\bC}{\mathbf{C}}
\newcommand{\bD}{\mathbf{D}}
\newcommand{\bE}{\mathbf{E}}
\newcommand{\bF}{\mathbf{F}}
\newcommand{\bG}{\mathbf{G}}
\newcommand{\bH}{\mathbf{H}}
\newcommand{\bI}{\mathbf{I}}
\newcommand{\bJ}{\mathbf{J}}
\newcommand{\bK}{\mathbf{K}}
\newcommand{\bL}{\mathbf{L}}
\newcommand{\bM}{\mathbf{M}}
\newcommand{\bN}{\mathbf{N}}
\newcommand{\bR}{\mathbf{R}}
\newcommand{\bS}{\mathbf{S}}
\newcommand{\bT}{\mathbf{T}}
\newcommand{\bU}{\mathbf{U}}
\newcommand{\bV}{\mathbf{V}}
\newcommand{\bW}{\mathbf{W}}
\newcommand{\bX}{\mathbf{X}}
\newcommand{\ba}{\mathbf{a}}
\newcommand{\bb}{\mathbf{b}}
\newcommand{\bc}{\mathbf{c}}
\newcommand{\bd}{\mathbf{d}}
\newcommand{\be}{\mathbf{e}}
\newcommand{\bff}{\mathbf{f}}
\newcommand{\bg}{\mathbf{g}}
\newcommand{\bj}{\mathbf{j}}
\newcommand{\bk}{\mathbf{k}}
\newcommand{\bm}{\mathbf{m}}
\newcommand{\bn}{\mathbf{n}}
\newcommand{\bp}{\mathbf{p}}
\newcommand{\br}{\mathbf{r}}
\newcommand{\bs}{\mathbf{s}}
\newcommand{\bt}{\mathbf{t}}
\newcommand{\bu}{\mathbf{u}}
\newcommand{\bv}{\mathbf{v}}
\newcommand{\bx}{\mathbf{x}}
\newcommand{\bl}{\mathbf{l}}
\newcommand{\bq}{\mathbf{q}}
\newcommand{\bw}{\mathbf{w}}
\newcommand{\by}{\mathbf{y}}
\newcommand{\bz}{\mathbf{z}}
\newcommand{\bZero}{\mathbf{0}}
\newcommand{\Lag}{\mathcal{L}}
\newcommand{\tran}{\mathsf{T}}
\newcommand{\dd}{\mathrm{d}}


\newcommand{\tbx}{\tilde{\mathbf{x}}}

% \newcommand{\Comment}[3]
% {
% \textcolor{#3}{#1: #2}
% }

\newcommand{\BT}[1]
{
\textcolor{teal}{BT: #1}
}


\newcommand{\revision}[1]
{
\textcolor{blue}{#1}
}

\newcommand{\TODO}[1]
{
\Comment{}{#1}{red}
}

\DeclareMathOperator*{\argmin}{argmin}
\DeclareMathOperator*{\argmax}{argmax}
\newcommand*{\argminl}{\argmin\limits}
\newcommand*{\argmaxl}{\argmax\limits}

\newif\ifcomment
%\commentfalse
\commenttrue
\newcommand{\LK}[1]{\ifcomment{\colorbox{orange}{\textcolor{white}{(LK: #1)}}}\else \fi}
\newcommand{\BB}[1]{\ifcomment{\colorbox{yellow}{\textcolor{white}{(BB: #1)}}}\else \fi}
\newcommand{\RF}[1]{\ifcomment{\colorbox{cyan}{\textcolor{white}{(RF: #1)}}}\else \fi}

% three new commands to highlight changes: \remove{} \add{} \replace{}{}
\newif\ifchanges
% \changestrue
\changesfalse
\newcommand{\remove}[1]{\ifchanges{\textcolor{red}{#1}}\else \fi}
\newcommand{\replace}[2]{\ifchanges{\textcolor{red}{#1}\textcolor{blue}{#2}}\else #2\fi}
\newcommand{\add}[1]{\ifchanges{\textcolor{blue}{#1}}\else #1\fi}
\newcommand{\removeTableRow}{\rowcolor{red}}

% paper notations
\newcommand{\numBS}{S} % F for frames
\newcommand{\numVertices}{N}
\newcommand{\numColumnB}{d}
\newcommand{\matA}{\mathbf{A}}
\newcommand{\approxMatA}{\mathbf{A'}}
\newcommand{\matB}{\mathbf{B}}
\newcommand{\matC}{\mathbf{C}}
\newcommand{\matM}{\mathbf{M}}
\newcommand{\vertexPos}{\mathbf{v}}
\newcommand{\bSWeight}{w}
\newcommand{\matBSWeight}{\mathbf{w}}
\newcommand{\numNonzeroBC}{K}
\newcommand{\numBitsPerNz}{Q}

\newcommand{\meanCurvature}{\mathbf{H}}
\newcommand{\wrinklesMap}{\mathbf{D}}
\newcommand{\wrinklesMult}{\gamma}
\newcommand{\restPose}{\mathbf{R}}
\newcommand{\vertexRestPose}{\mathbf{r}}
\newcommand{\identity}{\mathbf{I}}
\newcommand{\matWB}{\mathbf{wB}}
\newcommand{\getNumNz}{\operatorname{nnz}}
\newcommand{\newNorm}{\eta}

\newcommand{\Laplacian}{\mathbf{L}}
\newcommand{\reals}{\mathbb{R}}
\newcommand{\code}[1]{\texttt{#1}} % for short inline code like '

\newcommand{\csfb}{\texttt{CSFB}\xspace}
\newcommand{\qmf}{\texttt{QMF}\xspace}
\newcommand{\mf}{\texttt{MF}\xspace}

%% EG style seems highly opposed to citet
% \newcommand{\citet}[1]{\cite{#1}}


% Document starts
\begin{document}
\setlength{\abovecaptionskip}{0.5ex}
\setlength{\belowcaptionskip}{0.5ex}
\setlength{\floatsep}{0.5ex}
\setlength{\textfloatsep}{0.5ex}

\title{QMF-Blend: Quantized Matrix Factorization for Efficient Blendshape Compression}

\author{Roman Fedotov}
\affiliation{%
  \institution{Meta Reality Labs}
  \country{USA}
}
\email{TODO}

\author{Brian Budge}
\affiliation{%
  \institution{Meta Reality Labs}
  \country{USA}
}
\email{bbudge@meta.com}

\author{Ladislav Kavan}
\affiliation{%
  \institution{Meta Reality Labs}
  \country{Switzerland}
}
\email{
lkavan@meta.com}


\begin{abstract}
In this paper, we introduce a state-of-the-art blendshape compression algorithm that significantly reduces storage requirements and computational complexity in facial animation. Our approach leverages large sparse matrix factorization and quantization to compress high-dimensional blendshape coefficients into a compact representation, preserving essential features and high-frequency geometric details. The proposed algorithm outperforms existing methods in terms of compression ratio, reconstruction quality, and computational efficiency. We demonstrate its effectiveness through extensive experiments on various animated face models, achieving compression factors of up to 100$\times$ over sparse blendshapes with minimal impact on quality. Our technique offers compression rates up to 4.6$\times$ better than the prior state-of-the-art while also improving approximation error and preserving features like wrinkles. Additionally, our runtime computation is up to 3$\times$ faster than state-of-the-art on CPU and 70\% faster than state-of-the-art on GPU, facilitating high-quality facial animation on low-powered computing platforms with limited resources.
\end{abstract}

% previous version used to prompt Metamate:
% In this paper, we present a novel state-of-the-art blendshape compression algorithm that significantly reduces the storage requirements and computational complexity of facial animation. Our approach leverages factorization of large sparse matrices and quantization to compress high-dimensional blendshape coefficients into a compact representation while preserving the essential features of the original data, especially high-frequency geometric details.  The proposed algorithm outperforms existing methods in terms of compression ratio, reconstruction quality, and computational efficiency. We demonstrate the effectiveness of our approach through extensive experiments on various animated face models, with compression factors well over 100x with minimal impact on quality. Our compression rates are about 4x better than the prior state-of-the-art while simultaneously also improving the approximation error. Our runtime computation is 2x to 3x faster and thus facilitates high quality facial animation especially on low-powered computing platforms with limited resources.

\setcopyright{acmlicensed}

\ccsdesc[500]{Computing methodologies~Animation}

\keywords{facial animation, blendshapes, skinning decomposition}

\begin{teaserfigure}
    % \includegraphics[width=4.4cm]{example-image-a}
    % \includegraphics[width=4.4cm]{example-image-a}
    % \includegraphics[width=4.4cm]{example-image-a}
    % \includegraphics[width=4.4cm]{example-image-a}
    \hfill
    \begin{minipage}[b]{3.5cm}
         \includegraphics[width=3.5cm]{jupiterOrig_w.png}
         \centering
         \subcaption*{Sparse BS, 6.4 MB} %1e-3 -> 6.4 MB , 1e-4 -> 9.8
    \end{minipage}
    \hfill
    \begin{minipage}[b]{3.5cm}
         \includegraphics[width=3.5cm]{jupiterDemB_w.png}
         \centering
         \subcaption*{Dem Bones, 1.6 MB}
    \end{minipage}
    \hfill
    \begin{minipage}[b]{3.5cm}
         \includegraphics[width=3.5cm]{jupiterSD0_w.png}
         \centering
         \subcaption*{\csfb, 302.3 KB}
    \end{minipage}
    \hfill
    % \begin{minipage}[b]{3.5cm}
    %      \includegraphics[width=3.5cm]{jupiterSD_w.png}
    %      \centering
    %      \subcaption{CSFB + our new norm }
    % \end{minipage}
    \begin{minipage}[b]{3.5cm}
         \includegraphics[width=3.5cm]{jupiterSFQ_w.png}
         \centering
         \subcaption*{\qmf (Ours), 65.4 KB }
    \end{minipage}
    \hspace{1.5cm}
    \caption{Comparison of blendshape compression algorithms (Jupiter model). Dem Bones (\cite{dembones}) and \csfb(\cite{kavan2024compressed}) do not preserve wrinkles.  Our method, \qmf, preserves wrinkles well, while also providing much higher compression ratio. }
    \label{fig:Teaser}
\end{teaserfigure}
\maketitle


\section{Introduction}

% Blurb on why it's important to compress blendshapes
Users are consuming 3D graphics experiences on ever-shrinking devices in order to maximize user satisfaction; the ultimate goal is an eyeglasses form factor with minimal weight and volume. However, weight and power limitations require us to rethink classical computer graphics techniques that have served us well in the previous era of personal computers and game consoles. In this paper we focus on blendshape-based facial animation and present a new blendshape compression algorithm which promises to produce facial models suitable for upcoming computing platforms.

% Most popular is to approximate blendshapes via a skinning decomposition, but that's not ideal: skinning was designing for large articulated deformations
The problem of blendshape compression is not new; high visual fidelity demands hundreds or even thousands of blendshapes. Since classical quantization or sparsification methods provide only moderate data reduction \cite{seo2011compression}, many recent systems instead approximate blendshapes via linear blend skinning decomposition \cite{james2005skinning, dembones}. However, skinning is primarily intended for full-body deformations with large joint rotations (e.g.,  arms), and is not ideal for subtle facial deformations handled via numerically small but visually important correctives \cite{lewis2014practice}. Attempts to accurately fit the subtle facial deformations with skinning typically require a relatively large number of non-zero skinning weights per vertex (8+), which can be problematic on resource-constrained platforms \cite{kavan2024compressed}.

In this paper we argue for a different, lower-level approach, based on quantized sparse matrix factorization. %If the input matrix of blendshape deltas is denoted as $\matA\in\reals^{3 \numBS \times \numVertices}$, our goal is to approximate this matrix via a matrix product of $\matB\in\reals^{3\numBS \times \numColumnB} $ and $\matC\in\reals^{\numColumnB \times \numVertices}$, where $\matB$ and $\matC$ are sparse and $\numColumnB$ is a hyper-parameter. 
This factorization is related to linear blend skinning (which can also be formulated as a special type of matrix factorization \cite{kavan2010fast}), but distinct in the following ways: we decidedly forgo the structure of 3D skinning transformations and skinning weights (with the associated partition-of-unity constraints and per-vertex-nonzeros limits). 
%the nonzero values can be distributed arbitrarily among $\matB$ and $\matC$ \LK{is this actually true or do we impose some per-row / per-column limits?} \RF{as part of wrinkles correction we introduce upper limit for number of non-zero values per column of matrix C (per vertex). If we don't use wrinkle corrections we don ton impose any per column limits}. 
Sparse factorization is a more elementary abstraction than skinning and, in practice, allows higher flexibility in capturing high-frequency details in the input shapes $\matA$. Specifically, our results show well over a 3$\times$ reduction of storage compared to skinning decomposition-based compression with equal or even slightly improved approximation accuracy 

Given the matrix $\matA$, we solve for sparse matrices $\matB$ and $\matC$, such that $\matB\matC = \matA$, using a projected Adam optimizer \cite{kingma2014adam}, where the projection corresponds to our desired sparsification. To avoid non-smooth, visually disturbing errors $\matA - \matB \matC$, previous skinning decomposition work usually adds Laplacian regularizers \cite{le2014robust,kavan2024compressed}. This works, but attenuates the high-frequency details of $\matA$ such as wrinkles. Instead, we propose a new loss function, and a new function to bias toward retaining data in visually important regions, leading to more accurate approximation of details such as wrinkles (\Figure{fig:Teaser}) 

Another important opportunity to further reduce memory presents itself in the quantization of non-zero coefficients of $\matB$ and $\matC$. We demonstrate that by quantizing the non-zeros of $\matB$ and $\matC$ to 8- or 10.66-bits (11, 11, and 10 bits packed into a 32-bit integer), we introduce only negligible error, but reduce memory even more to achieve an overall 3$\times$ to 5$\times$ storage savings over a state-of-the-art skinning decomposition method with sparse skinning transformations \cite{kavan2024compressed} while simultaneously also \textit{improving} the approximation accuracy. For these reasons we argue that quantized sparse matrix factorization is an important technique for delivering high quality blendshape animation on resource-constrained platforms.

Our key contributions are that we:
\begin{itemize}
    \item depart from skinning decomposition approaches and introduce a new way to compress blendshapes via factorization into sparse matrices;
    \item propose a new loss function and an additional technique that combine to preserve high-frequency details like wrinkles;
    \item and demonstrate quantization of the non-zero matrix values further to reduce memory while preserving sufficient precision.
\end{itemize}

\vspace{-1.5mm}
\section{Related Work}

Unsupervised machine learning methods such as Lasso \cite{tibshirani1996regression}, non-negative matrix factorization \cite{lee1999learning,hoyer2004non} and overcomplete dictionary learning \cite{aharon2006} discover sparse structures in input data. The theoretical and algorithmic aspects of sparse matrix factorization have been also studied in depth \cite{bach2008convex,neyshabur2013sparse,gribonval2010dictionary}. Quantization of low-rank matrix approximation has also been recently considered in the context of compression \cite{saha2023matrixcompressionrandomizedlow}. Our specific case of blendshapes, however, presents particular geometric challenges, since high visual quality of the resulting facial animations is one of our main goals.

In computer animation, the traditional Facial Action Coding System (FACS) \cite{ekman1978facial} and its variants have long been the foundation of facial blendshape models. However, these systems are limited by their complex controls and requirement for a large number of blendshapes to achieve high fidelity \cite{kim2021optimizing,choi2022animatomy}. 
In cases where training data or facial rigs are available, classical statistical methods \cite{meyer2007key} and deep learning approaches \cite{bailey2020fast,chandran2022facial,dafei2023neural} can be employed to derive more efficient models. Our approach does not require any neural networks, even though our quantized matrix factorization utilizes the same underlying techniques such as the popular Adam optimizer \cite{kingma2014adam}.

Blendshape compression is also related to the problem of skinning decomposition, first introduced by \citet{james2005skinning} in the context of linear blend skinning. Subsequent research led to improvements in approximating the global optimum, starting with the addition of bone rigidity constraints by \cite{kavan2007skinning} or fitting input animations with hierarchical skeletons, as demonstrated by \citet{schaefer2007example}. Later, \citet{kavan2010fast} proposed an alternating solver achieving better approximations of the unknown global optimum, while \citet{hasler2010learning,bharaj2012automatically} further advanced the utilization of hierarchical skeletons. The benefits of adding bone rigidity constraints were further studied by \citet{le2012smooth}, which formed the basis for the popular Dem Bones implementation \cite{dembones}. Beyond classical skinning methods, \citet{seo2011compression} investigated blendshape compression via hierarchical semi-separable matrices and \citet{le2013two} proposed two-layer skinning for compressing skinning weights. High-quality jaw opening can be achieved without skeletons using "intermediate" blendshapes, as shown by \citet{lewis2014practice}, or with hierarchical skeletons, as demonstrated by \citet{tianye2017flame}. Finally, GPU-specific methods have been proposed to speed up blendshape evaluation using compute shaders \cite{costigan2016improving}, and deep learning-based approaches have been explored for fitting hierarchical skeletons \cite{moutafidou2023deep}.  Our primary point of comparison for the paper will be a recent state-of-the-art skinning decomposition-based compression method: Compressed Skinning for Facial Blendshapes (\csfb) \cite{kavan2024compressed}. Our proposed matrix factorization trades the traditional skinning formulation for higher performance.

Blendshapes are not the only way to animate faces; recent advances in anatomically-inspired systems and implicit neural representations have shown great promise in modeling complex geometries, including the human face. Notably \citet{choi2022animatomy} presented Animatomy, a novel representation of the human face that uses muscle fiber curves as an anatomical basis for fine-grained parameterization of facial expressions. \citet{Chandran_2024_CVPR} proposed Implicit Face Models, which learn to jointly model the facial anatomy and skin surface using an ensemble of implicit neural networks. However, blendshape models are orders of magnitude computationally less expensive, and so real-time systems on resource-constrained platforms continue to rely on blendshapes.



\section{Method}
A classical blendshape model comprising $\numVertices$ vertices and $\numBS$ blendshapes, can be formulated using the rest pose mesh and "deltas" between the rest pose and individual blendshapes \cite{lewis2014practice}:
\begin{equation}
\vertexPos_{0,i} + \sum_{k} \bSWeight_k \Delta\vertexPos_{k,i},
\label {eq:ShapeCalculation}
\end{equation}
where $\vertexPos_{0,i}\in\reals^3$ is the $i$-th rest-pose vertex, $\Delta\vertexPos_{k,i}$  denotes the positional difference between the $i$-th vertex of the $k$-th blendshape and its corresponding vertex in the rest pose, and
$\bSWeight_k$ are blendshape weights defining facial expression; in this paper, we treat these as inputs. In typical runtime systems, the $\bSWeight_k$ are computed via facial rigs which may also contain nonlinearities \cite{lewis2014practice}, but the subsequent blending is linear. We can stack all of the blendshapes into a matrix $\matA\in\reals^{3\numBS \times \numVertices}$:
\[
\matA =
\begin{bmatrix}
  \Delta\vertexPos_{1,1} & \dots & \Delta\vertexPos_{1,\numVertices} \\
  \vdots & \ddots & \vdots \\
  \Delta\vertexPos_{\numBS,1} & \dots & \Delta\vertexPos_{\numBS,\numVertices} \\
\end{bmatrix}
\]
Modern models often consist of hundreds to thousands of blendshapes and thus the storage of the matrix $\matA$ is an important consideration. The runtime speed of facial expression calculation is primarily determined by fetching elements of $\matA$ from memory.
The main objectives of our method is to develop an efficient representation for $\matA$ to enhance the real-time performance of facial expression calculations while simultaneously reducing the memory footprint.

Our method represents matrix $\matA$ as the product of two sparse matrices, $\matB\in\reals^{3S\times \numColumnB}$ and $\matC\in\reals^{\numColumnB\times \numVertices}$, where $\numColumnB$ is a hyper parameter. The problem of sparse factorization of matrix $\matA$ can be formulated as the following minimization problem:
%\begin{equation}
%  \min \| \matA - \matB\matC \|_p
%    \label{eq:Minimization} % todo add "subject to:" ?
%\end{equation}
\begin{equation}
  \min_{\matB, \matC} \| \matA - \matB\matC \|,
    \label{eq:Minimization} % todo add "subject to:" ?
\end{equation}
subject to the sparsity constraints on $\matB$ and $\matC$. 
%The error norm is typically the Frobenius norm or its higher-order powers \cite{kavan2024compressed}. 
%These standard norms work well, but in \Section{sec:newNorm} we will introduce a norm that better preserves high frequency details.
%We do not optimize the rest pose positions $\vertexPos_{0,i}$.
%, as any imperfection in this position will affect quality of every blendshape so we assumed it to be given data. \LK{I didn't understand the previous sentence...}
%We have implemented \Equation{eq:Minimization} in PyTorch. The only constraints we apply on this stage was total number of non-zero values in both matrices $\matB$ and $\matC$. 
%---
%Our loss function primarily utilizes the standard Euclidean norm, corresponding to $p=2$\LK{this should be discussed earlier, right after the minimization problem equation}, which eliminates the need to calculate absolute values in \Equation{eq:Minimization}. To ensure a smooth approximation, we employed our improved loss function (\Eq{eq:NewLoss}).

To find solutions for \Equation{eq:Minimization}, we adopt the approach from \cite{kavan2024compressed}, which combines the Adam optimizer with a projection step to enforce sparsity of both matrices $\matB$ and $\matC$. These projections do not participate in automatic differentiation.
%, just like with general proximal algorithms \cite{parikh2014proximal}. In our case, 
Each projection consists of retaining only $\numNonzeroBC$ total number of non-zero values across both matrices $\matB$ and $\matC$. We tried other projection strategies, such as setting separate limits for the numbers of non-zero values for each matrix $\matB$ and $\matC$. However, simply retaining the $\numNonzeroBC$ largest numbers across both of the matrices gives the optimizer the most flexibility and yields the best results. Our method can allocate the non-zero values better than skinning decomposition methods; areas with more complex deformation receive a higher number of non-zero values.

\subsection{Addressing missing wrinkles}\label{sec:newNorm}
We found that both Dem Bones as well as \csfb lose wrinkles at medium to high compression rates (see \Figure{fig:Teaser}). This is because these previous works apply a Laplacian regularizer to ensure the approximation error is smooth:
\begin{equation}
 \min_{\matB, \matC}( \|\matA - \matB \matC \| + \alpha \|\Laplacian (\matB \matC)^\tran \| ),
 \label{eq:OldLoss}
\end{equation}
where $\Laplacian$ is the graph Laplacian.
The Laplacian term penalizes large curvatures in $\matB \matC$ which leads to smooth error, but also smooths out the original signal, because the Laplacian is zero only for affine functions. This is especially problematic in areas where the original blendshapes $\matA$ have a large curvature, which is undesirably penalized by the regularization term, leading to over-smoothing of visually important details.

To address these issues, we propose to replace the $\alpha \|\Laplacian (\matB \matC)^\tran\|$ with the following term: %loss that avoids the problems of the Laplacian regularizer: $\newNorm(\matA - \approxMatA)$, where 
\begin{equation}
\alpha \|\Laplacian(\matA - \matB \matC )^\tran \|.
%\newNorm(\matM) = \| \matM \| + \alpha\|\Laplacian\matM\|
%\|\Laplacian(\approxMatA -\matA)\|
  \label{eq:NewLoss}
\end{equation}
%
The seemingly minor change of applying the Laplacian to $(\matA - \matB \matC)^\tran$ means that the original blendshapes (i.e., the exact reconstruction: $\matA = \matB \matC$) would now receive zero regularization penalty -- even in high-curvature areas, avoiding the oversmoothing issues of the previous approach. We can say that \Equation{eq:OldLoss} makes a smooth mesh while \Equation{eq:NewLoss} makes a smooth error.

\begin{comment}
We have incorporated a new norm $\newNorm$ into \csfb \LK{this may be better discussed in Results, here we should probably focus on our new method and not digress into the potential improvements of CSFB...}. This approach yields visual improvements in areas with wrinkles, as demonstrated in \Figure{fig:CSFBAura}. Moreover, our results show enhanced quality for the models, leading to overall better performance of the skinning decomposition.
\end{comment}

% Optimization

\begin{figure}
  \parbox{0.32\linewidth}{
    \includegraphics[width=\linewidth]{numNzBefore.png}
    \subcaption{}
    \label{nonZeroDistribution}
  }
  \parbox{0.32\linewidth}{
    \includegraphics[width=\linewidth]{numNzMap.png}
    \subcaption{}
    \label{wrinklesMap}
  }
  \parbox{0.32\linewidth}{
    \includegraphics[width=\linewidth]{numNzAfter.png}
    \subcaption{}
    \label{nonZeroDistributionAfter}
  }
\captionsetup{subrefformat=parens}
\caption{Sparse Factorization:
\subref{nonZeroDistribution} Number of non-zero values in $\matC$ per vertex;
\subref{wrinklesMap} normalized wrinkles density $\wrinklesMap$;
\subref{nonZeroDistributionAfter} number of non-zero values in $\matC$ after correction.}
    \label{fig:FloatsDistribution}
\end{figure}


\subsubsection{Further wrinkle improvements for sparse matrix factorization}
While our new regularization term helps preserve the geometric error, high compression rates still lead to some degradation and over-smoothing. In this section we describe an additional technique we used to even better preserve the shape of high-frequency details like wrinkles.
Unlike skinning decompositions methods such as \csfb, which use a fixed number of weights per vertex, sparse matrix factorization enables non-zero values to appear in the matrix where they will have the most impact to reduce overall loss (\Subfig{fig:FloatsDistribution}{nonZeroDistribution}). However, we found that it is beneficial to explicitly enforce more non-zeros and hence higher accuracy in high-curvature areas.
Our initial hypothesis was that the optimization process would smooth out all deformations with high mean curvature; however, this proved to be incorrect. Instead, we found that the lack of detail preservation arose in areas where changes in mean curvature occurred only \emph{across a limited number of blendshapes} because the blendshapes themselves are used when calculating loss, and values appearing in few blendshapes will have less contribution in the loss (see \Figure{fig:QMFNewNormAndWrinkles}). In such cases the optimizer reduced the number of non-zero values per vertex making it impossible to reconstruct high curvature deformations. 
%-------
To identify these areas, we analyze the changes in mean curvature across all blendshapes.
For the $i$-th vertex, we assess how its curvature varies across blendshapes by calculating the differences between its curvature in the rest pose and in each blendshape.
$\Delta\meanCurvature_i = (|\meanCurvature_{1,i} - \meanCurvature_{0,i}|, \dots, |\meanCurvature_{S,i} - \meanCurvature_{0,i}|)^\tran$, where $\meanCurvature_{k,i}$ is mean curvature of $i$-th vertex in $k$-th blendshape. 
To find vertices where curvature only changes in a few blendshapes we use
\[
\wrinklesMap_i = \max_j(\Delta\meanCurvature_{i,j})/ \overline{\Delta\meanCurvature^l_i},
\]
where $\overline{\Delta\meanCurvature^l_i}$ is the average of mean curvature "deltas"  for vertex $i$ excluding the $l$ largest values, with $l$ typically 5.
\Subfigure{fig:FloatsDistribution}{wrinklesMap} shows a visualization of this metric.
Since these calculations are performed only once, prior to the optimization process, we can afford to calculate the mean curvature accurately using the cotangent Laplacian matrix for each blendshape.
%
\begin{figure}
    \includegraphics{wrinkles.pdf}
    \caption{Wrinkle density calculation and QMF Results (left to right): result with original loss function; result with new loss (wrinkles on right side only); result with new loss and wrinkle map (wrinkles on both sides).}
    \label{fig:QMFNewNormAndWrinkles}
\end{figure}
%

Once the winkles distribution is calculated, we can use it in optimization by multiplying the $i$-th column of matrix $\matC$ by $\wrinklesMult\wrinklesMap_i + 1$ before culling the values in matrices $\matB$ and $\matC$. The $\wrinklesMult$ parameter is chosen for each model to be the smallest value that preserves the original wrinkle areas.
A situation could arise where several vertices have too many non-zero values, and that could degrade the quality of other areas.  To address this we limited the maximum number of non-zero values per vertex to 30. We can see the result of this corrected optimization in \Figure{fig:QMFNewNormAndWrinkles}.
The calculation of the matrix $\wrinklesMap$ occurs once prior to the optimization, thus having negligible overhead.
% \begin{figure}
%     \includegraphics[width=0.31\linewidth, trim={3cm 29cm 12cm 19cm},clip]{evolutionSF_jupiter_SF_OldL.png}
%     \includegraphics[width=0.31\linewidth, trim={3cm 29cm 12cm 19cm},clip]{evolutionSF_jupiter_SF_L.png}
%     \includegraphics[width=0.31\linewidth, trim={3cm 29cm 12cm 19cm},clip]{evolutionSF_jupiter_SF_L_W.png}
%     \caption{QMF results left to right: Result with original loss function, new loss (wrinkles on right side only), new loss and wrinkles map (wrinkles on both sides)}
%     \label{fig:QMFNewNormAndWrinkles}
% \end{figure}
\subsection{Compression of sparse matrices}
To efficiently encode indices for the sparse matrices $\matB$ and $\matC$, we utilize a modified compressed sparse row (CSR) format for matrix $\matB$ and a compressed sparse column (CSC) format for matrix $\matC$.   
These choices are determined by the order of matrix multiplication at runtime. Specifically, for matrix $\matB$, we need to compute a weighted sum of its rows. In contrast, matrix $\matC$ serves as the second operand in the matrix multiplication, so we must access its elements column-wise (see \Section{sec:runtime}).

Experimental results indicate that  $\numColumnB = 200$ (columns of $\matB$ and rows of $\matC$) is sufficient to achieve low errors. Because $\numColumnB \le 255$, we use an 8-bit representation to encode the column index for matrix $\matB$ and the row index for matrix $\matC$. Instead of storing an array of offsets, we store the number of non-zero values per row in matrix $\matB$ or per column in matrix $\matC$ and calculate offsets in order to calculate the next dot product during matrix multiplication. 
%For example, see how \texttt{numValuesRow} is used in \Algorithm{alg:RuntimeCode}. 
Consequently, for indexing both matrices, we need to store $\numVertices + 3\numBS + \numNonzeroBC$ bytes for $\numVertices$ vertices, $\numBS$ blendshapes, and $\numNonzeroBC$ non-zero values.
%
%--- Quantization
\begin{figure}
    % TODO make better image
    \includegraphics[width=\linewidth]{quantizationMetricSlim.pdf}
    \caption{Quantization metrics.}
    \label{fig:QuantizationMetrics}
\end{figure}
\begin{figure}
    \includegraphics[width=0.48\linewidth, trim={1cm 1cm 1cm 7cm},clip]{images/no_noise.png}
    \includegraphics[width=0.48\linewidth, trim={1cm 1cm 1cm 7cm},clip]{images/noise.png}
    \caption{Quantization noise: Left: original floating-point blendshape (Louise model); right: blendshape after 8-bit quantization.}
    \label{fig:quantizationNoise}
\end{figure}
%

Having significantly reduced the data for indexing the sparse matrices, what remains is to reduce the amount of memory used to encode the floating point non-zero values. We employ quantization to minimize the number of bits required to represent each value.
After optimization is complete, we independently quantize the non-zero values for both matrices $\matB$ and $\matC$. For each of  $\matB$ and $\matC$, we find the minimum and maximum scalar values.  Then we use the following quantization formula for $Q$ bits:
\begin{equation*}
  V_Q = \text{round}((V - V_{\mathrm{min}})/(V_{\mathrm{max}} - V_{\mathrm{min}}) \cdot (2^Q - 1))
\end{equation*}

We store $V_{\mathrm{min}}$, $V_{\mathrm{max}}$, and $Q$, and this allows us to dequantize via: 
\begin{equation*}    
  V = V_{\mathrm{min}} + (V_Q \cdot (V_{\mathrm{max}} - V_{\mathrm{min}}) )/(2^Q - 1)
\end{equation*}


%
As shown in \Figure{fig:QuantizationMetrics}, our metrics degrade gradually as we reduce the number of bits per value. Notably, the data indicates that using 8-bit quantized values only slightly compromises the final quality of the shape, seeming to make it a viable option for optimizing memory usage.  However, despite the positional metrics' accuracy, we were able to discern visible noise with larger deformations on our denser models (see \Figure{fig:quantizationNoise}).  This quantization noise corresponds to areas where a small error will have a noticeable effect on local curvature.
To address this, we employ the average edge angular difference metric (EAD, detailed in \Section{sec:results}) to compare the quantized matrices with their floating-point counterparts. We select a quantization factor that minimizes degradation of this metric, ensuring it remains within a few percent of the original value.

The quantization levels  we have tried are 8-bit, 16-bit, and 10.66-bit.  10.66-bit is using an encoding of $\{11, 11, 10\}$ bits per 32-bit integer.  In our results, 8-bit quantization was sufficient for our lower-polygon models, and 10.66-bit provided sufficient quality for all models.  16-bit quantization may still prove useful in very polygon-dense models, or models with very extreme blendshape animation.  Additionally, even for models requiring only 10.66-bit quantization, 16-bit quantization provides an alternative that enables faster runtime reconstruction. 

\begin{figure}
    % TODO make better image
    \includegraphics[width=\linewidth]{lossHistorySlim.pdf}
    \caption{Final loss depends on random seed initialization.}
    \label{fig:InitErrors}
\end{figure}

\subsection{Initialization of the optimizer}
The results obtained with our method are somewhat dependent on random initialization (a trait shared with \csfb).  As can be seen in \Figure{fig:InitErrors}, 
the difference in loss could be about 15\% between choosing good and poor random seeds.  We observed, however, that once started, the optimization trajectories for the best seeds, seem to be somewhat predictable after about 5\% of iterations.  Therefore, a practical approach to achieve near-optimal results is to run several hundred iterations with several seeds and select the best thus far to continue optimizing.  In practice, we use 20 random seeds running 5\% of iterations, and then run the best one to 100\%, reducing the runtime cost to a good solution by a factor of about 10x.

\subsection{Runtime implementation} \label{sec:runtime}

\begin{comment}

\begin{lrbox}{\codebox}
\begin{lstlisting}[language=c++, basicstyle=\scriptsize]
template <typename BlendShapeData, typename ValueType>
void computeSparseFactorizationBlendshapes(
    const BlendShapeData& data,
    DeviceSpan<const Vec3>& staticMeshCoords,
    DeviceSpan<Vec3>& outMeshCoords) {
  constexpr bool isQuantized = !std::is_floating_point_v<ValueType>;
  float bMult = data.bRange / std::numeric_limits<ValueType>::max();
  float cMult = data.cRange / std::numeric_limits<ValueType>::max();

  std::vector<float> matrixWB(data.numColumnB * 3lu, 0.f);
  uint32_t flatBIndex = 0;
  for (uint32_t bsIndex = 0; bsIndex < data.numBlendshapes; ++bsIndex) {
    float bsWeight = data.blendshapeWeights[bsIndex];
    for (uint32_t i = 0; i < 3; ++i) {
      uint32_t numValuesRow = data.matrixBNumValuesRow[bsIndex * 3 + i];
      for (uint32_t j = 0; j < numValuesRow; ++j) {
        float bValue = isQuantized ?
          data.bMin + bMult * data.matrixB[flatBIndex] : 
          data.matrixB[flatBIndex];
        uint32_t bCol = data.matrixBIndices[flatBIndex];
        matrixWB[3 * bCol + i] += bValue * bsWeight;
        flatBIndex++;
      }
    }
  }
 
  size_t numVertices = outMeshCoords.size();
  uint32_t flatCIndex = 0;
  for (uint32_t vIndex = 0; vIndex < numVertices; ++vIndex) {
    Vec3 res(0.f);
    uint_fast16_t numValuesCol = data.matrixCNumValuesCol[vIndex];
    for (uint_fast16_t i = 0; i < numValuesCol; ++i) {
      uint32_t cRow = data.matrixCIndices[flatCIndex + i];
      float cValue = isQuantized ?
        data.cMin + cMult * data.matrixC[flatCIndex + i] : 
        data.matrixC[flatCIndex + i];
      res.x += matrixWB[3 * cRow + 0] * cValue;
      res.y += matrixWB[3 * cRow + 1] * cValue;
      res.z += matrixWB[3 * cRow + 2] * cValue;
    }
    flatCIndex += data.matrixCNumValuesCol[vIndex];
    outMeshCoords[vIndex] = staticMeshCoords[vIndex] + res;
  }
}
\end{lstlisting}
\end{lrbox}

\begin{algorithm}
\caption{CPU code to apply sparse matrix factorization-based blendshapes at runtime.}
\label{alg:RuntimeCode}
\usebox{\codebox}
\end{algorithm}

\end{comment}

Calculation of the \Equation{eq:ShapeCalculation} with sparse representation of matrix $\matA$ requires two matrix multiplications and a vector addition:
\[
 \bSWeight \matB \matC + \restPose,
\]
where $\bSWeight = (\identity \bSWeight_1, \dots, \identity \bSWeight_\numBS)$, $\identity$ is a $3\times3$ identity matrix and $\restPose\in\reals^{3\times\numVertices}$ are vertex positions of the rest pose. The result of the first calculation is a dense matrix $\bSWeight \matB\in\reals^{3\times\numColumnB}$. We allocate the column-major matrix $\matWB$ to hold $3\times\numColumnB$ float numbers and initialize it with zeros.
%\BB{TODO, Roman, please fix?}
\begin{comment}
{\tiny
\begin{equation}
\begin{NiceArray}{ccccccccc}[cell-space-limits=2pt]
                &                         &   &   &        &                     & \color{red} x   & \Block{7-2}<\Large>{B} & \hspace*{0.1cm}\\
                &                         &   &   & w_0    & \times blendshape_0 & \color{green} y &                        & \hspace*{0.1cm}\\
 \color{red}  x & \Block{3-1}<\Large>{wB} &   & + &        &                     & \color{blue} z  &                        & \hspace*{0.1cm}\\
 \color{green}y &                         & = &   &        &                     & \color{red} x   &                        & \hspace*{0.1cm}\\
 \color{blue} z & \hspace*{0.7cm}         &   &   & w_1    & \times blendshape_1 & \color{green} y & \hspace*{0.5cm}        & \hspace*{0.1cm}\\
                &                         &   & + &        &                     & \color{blue} z  &                        & \hspace*{0.1cm}\\
                &                         &   &   & \vdots &                     & \vdots          &                        & \hspace*{0.1cm}\\
                &                         &   &   &        &                     &                 &                        & \hspace*{0.1cm}\\
\CodeAfter
\SubMatrix[{3-2}{5-2}]
\SubMatrix[{1-5}{7-5}]
\SubMatrix[{1-8}{7-9}]
% \SubMatrix[{9-5}{11-6}]
\emph{\SubMatrix{\{}{1-7}{3-7}{.}}
\emph{\SubMatrix{\{}{4-7}{6-7}{.}}
  % \tikz \draw [black] (8.5-|1) -- (8.5-|last) ;
\end{NiceArray}
\label{eq:wB}
\end{equation}
}
\end{comment}
To perform this multiplication efficiently we iterate over the array $\bSWeight$. For every non-zero weight value we need to iterate over 3 rows of the matrix $\matB$. If the matrix $\matB$ has non-zero values for a given weight, we perform multiplication and store the result in $\matWB$.
The final step of the algorithm is to multiply the small dense matrix $\matWB$ with $\matC$. We iterate over the columns of the matrix $\matC$, and for every non-zero value we will read a column in matrix $\matWB$. We add the result of multiplication to the value from the rest pose and store the result. 
%The CPU source code is available in \Algorithm{alg:RuntimeCode}.


\begin{lrbox}{\codebox}
\begin{lstlisting}[language=c++]
template <typename T>
__global__ void computeWB(float* matrixWB, const float* weights,
    const uint16_t* B_perValueLookupInfo, const T* matrixB, 
    const uint8_t* matrixBIndices, float bMin, float bMult,
    int32_t numColB, int32_t numBValues) {
  constexpr bool isQuantized = !std::is_floating_point_v<T>;
  int flatBIndex = blockIdx.x * blockDim.x + threadIdx.x;
  if (flatBIndex >= numBValues) { return; }

  uint16_t lookupVal = B_perValueLookupInfo[flatBIndex];
  int bsIndex = lookupVal & 0x3fff;
  int i = lookupVal >> 14;
  float bsWeight = weights[bsIndex];

  float bValue = isQuantized ? 
    bMin + matrixB[flatBIndex] * bMult : 
    matrixB[flatBIndex];
  uint32_t bCol = matrixBIndices[flatBIndex];
  atomicAdd(&(matrixWB[3 * bCol + i]), bValue * bsWeight);
}

template <typename T, typename OffT>
__global__ void computeWBC(Vec3* outMeshCoords, 
    const Vec3* staticMeshCoords, const float* matrixWB, 
    const OffT* C_offsets, const uint8_t* matrixCIndices, 
    const T* matrixC, float cMin, float cMult, int32_t numColB) {
  constexpr bool isQuantized = !std::is_floating_point_v<T>;
  int vIndex = blockIdx.x * blockDim.x + threadIdx.x;
  if (vIndex >= outMeshCoords.size()) { return; }

  Vec3 res(0.0f);
  int nextIndex = C_offsets[vIndex + 1];
  int flatCIndex = C_offsets[vIndex];
  while(flatCIndex < nextIndex) {
    uint32_t cRow = matrixCIndices[flatCIndex];
    const float cValue = isQuantized ? 
        cMin + matrixC[flatCIndex] * cMult : 
        matrixC[flatCIndex];

    res.x += matrixWB[3 * cRow + 0] * cValue;
    res.y += matrixWB[3 * cRow + 1] * cValue;
    res.z += matrixWB[3 * cRow + 2] * cValue;
    ++flatCIndex;
  }
  outMeshCoords[vIndex] = staticMeshCoords[vIndex] + res;
}

\end{lstlisting}
\end{lrbox}

\begin{algorithm}
\caption{CUDA code to apply sparse matrix factorization-based blendshapes at runtime.}
\label{alg:CudaCode}
\usebox{\codebox}
\end{algorithm}

Efficiently computing this reconstruction on the GPU requires some modification, including additional memory.  Notably, in order to compute $wB$, we precompute a per-nonzero-$\matB$-value lookup to see which blendshape weight to use, and whether the value corresponds to $x$, $y$, or $z$.  This data is packed into 16-bits per value in $\matB$, and allows us to compute $\matWB$ in parallel across non-zero $\matB$ values.  Additionally, in order to compute $\matWB \matC$, we precompute a 16- or 24-bit offset per vertex to assist with indexing into $\matC$ (with number of bits depending on how large $\matC$ is).  This enables us to run in parallel across vertices. \Algorithm{alg:CudaCode} shows the our kernel code.

\section{Results}\label{sec:results}
Our results are captured on a Jetson Orin Nano 8GB, which provides a 6-core Arm Cortex-A78AE, a 1024 CUDA-core GPU, and 8 GB of  128-bit LPDDR5 RAM.   We run the Jetson in \SI{15}{\watt} mode. For our CPU results, we use only a single CPU core for runtime animation.  We demonstrate our technique with 6 blendshape datasets.  \Table{tab:FullResults} and \Figure{fig:Datasets} show our datasets.  Aura, Bowen, Jupiter, and Proteus are publicly available models \cite{kavan2024compressed}.  Louise is the Animatable Digital Double of Louise by Eisko© ( \href{http://www.eisko.com}{http://www.eisko.com} ).  We have also created a new face model, Carlos.
In our experiments, we use a fairly traditional sparse blendshapes format as a baseline.  $\matA$ is stored as an array of blendshape where each blendshape is an array of vertex indices and a corresponding array of non-zero deltas.  As part of our preprocessing pipeline, deltas smaller than $10 \mu m$ in absolute value are clamped to zero. 
The vertex indices for the sparse blendshape representation are 32-bit integers, and each non-zero delta consists of three 32-bit floating point values.

We compare our method against \csfb. For each dataset we use 400 bones, 8 weights per vertex, and 6,000 non-zero values for transformations, except the Louise and Carlos datasets, where 20,000 non-zero values were required for high-quality results. 
The chosen number of bones is higher than in \cite{kavan2024compressed} to allow \csfb better capture small details, while maintaining the same number of non-zeros for transformation matrices. This ensures a fair quality comparison with our matrix factorization methods.
We first ran experiments to see the effect of applying our new loss function (\Eq{eq:NewLoss}) on \csfb.  We observed significant quality improvements in wrinkle areas  (see \Figure{fig:CSFBAura}) and slight improvements in metrics (see \Table{tab:FullResults}).  Having established that our new loss function produces better results, the remainder of comparisons against our sparse matrix factorization technique will be using \csfb with the new loss function.

\begin{figure}
    \includegraphics[width=\linewidth]{chooseAlpha.pdf}
    \caption{Choosing $\alpha$ (Bowen model).}
    \label{fig:chooseAlpha}
\end{figure}

To demonstrate the synergistic effect of the new loss function with the wrinkles map, we created a synthetic dataset by modifying the Jupiter dataset so that every second blendshape is a copy of the blendshape "right eyebrows up".   In \Figure{fig:QMFNewNormAndWrinkles}, we can see (a) that with the original loss function (\Eq{eq:OldLoss}), we are never able to recover wrinkles.  Additionally, we can see (b) that by applying the new loss function (\Eq{eq:NewLoss}), we are able to recover wrinkles on the right side of the face, while we continue to lose them on the left side of the face.  (c) shows that by applying both the new loss function and wrinkles map, we reliably recover both sets of wrinkles. While not shown explicitly in the figure, using the wrinkle map with the old loss function leads to no discernible change from (a).

To measure the quality of optimization and to compare algorithms we use three metrics.  The first two are
mean absolute error (MAE) and maximum absolute error (MXE), corresponding to $L^1$ and $L^\infty$ norms.
MAE is the mean over all vertices of $|\matA - \matB\matC|$ and $\mathrm{MXE} = \max{|\matA - \matB\matC|}$ over all vertices.  The third metric attempts to measure noise introduced by optimization and quantization, and we use the average edge angular difference (EAD). This is calculated as the mean of $ | \beta_{A} - \beta_{BC} | $  over all edges, where $\beta$ represents the dihedral angle between an edge's adjacent faces.

To determine the optimal value of parameter $\alpha$, we conducted multiple optimization runs for all models and plotted the resulting curves for our metrics. Notably, all models exhibited a consistent behavior: as
$\alpha$ increased, the noise metric improved rapidly, while the position metrics degraded almost linearly (see \Figure{fig:chooseAlpha}) For every model we choose the biggest $\alpha$ value that did not significantly degrade the position metrics.

%----
To compare \csfb with sparse matrix factorization, we selected parameters that achieved the same or slightly better quality of optimization, allowing us to compare sizes and runtime costs. 
The number of non-zero values was chosen within a range of 35,000 to 140,000, depending on the level of detail in the model and the complexity of the blendshapes.
The value of $\wrinklesMult$ was selected from a range of 0.0006 to 0.003.
As shown in \Figure{fig:Teaser} and  \Figure{fig:QMFNewNormAndWrinkles}, the sparse factorization method preserves details in challenging wrinkle areas. 
%
 The resulting compression rates are illustrated in \Figure{fig:Compression}.  \qmf demonstrates a 3.2$\times$ to 4.6$\times$ higher compression rate than \csfb.

To assess the differences between algorithms in a real-time environment, we implemented sparse blendshapes, \csfb, and sparse matrix factorization in C++ and in CUDA. The \csfb implementation was highly optimized and works directly with a sparse matrix of bone transformations with only 6 degrees of freedom for each bone.  Like \qmf, sparse blendshapes required some auxiliary precomputed data for efficient GPU computation. \csfb does not require extra information for efficient processing on the GPU.

We executed the same animation sequence with precomputed blendshape weights with each algorithm and measured average wall clock time, with results shown in \Figure{fig:RuntimeCost}. \qmf selects 8-bit quantization for the Aura and Jupiter models, and 10.66-bit as the smallest quantization and 16-bit as the fastest runtime quantization for all other models. 
We can observe that our sparse factorization methods are significantly faster than an optimized version of \csfb. \qmf can be slightly slower than non-quantized sparse matrix factorization due to additional costs for dequantization and because the non-quantized sparse matrix factorization data already fits in CPU cache.  

When evaluating on the GPU, we noted that a common scenario would be to animate multiple faces per frame. To that end, we run 10 animations concurrently.  Results for our CUDA implementation are shown in \Figure{fig:RuntimeCostGPU}.  We note that
10.66-bit quantization incurs a fairly heavy runtime penalty of about 30\% on the CPU, but a smaller penalty up to about 15\% on the GPU.

Detailed results for all experiments are shown in \Table{tab:FullResults}.   Full optimization code for our method will be released open source concurrent to the conference. 


\begin{figure}
    \includegraphics[width=\linewidth]{durationDiagramCPU.pdf}
    \caption{CPU: Average facial expression calculation wall time}
    \label{fig:RuntimeCost}
\end{figure}

\begin{figure}
    \includegraphics[width=\linewidth]{durationDiagramCUDA.pdf}
    \caption{GPU: Average wall time for 10 concurrent facial expression calculations}
    \label{fig:RuntimeCostGPU}
\end{figure}



    \begin{table*}[t]
    \begin{center}
    \begin{tabular}{ l  r r r r r r }
\hline
  & Aura & Jupiter & Bowen & Proteus & Louise & Carlos\\
\hline
num vertices & 5944 & 5944 & 23735 & 23735 & 20053 & 23735\\
num BS & 267 & 319 & 253 & 287 & 253 & 287\\
dense BS & 18.2 MB & 21.7 MB & 68.7 MB & 78.0 MB & 58.1 MB & 78.0 MB\\
\multicolumn{6}{c}{\textbf{Sparse Blendshapes}}\\
size & 6.4 MB & 6.4 MB & 24.4 MB & 25.0 MB & 24.7 MB & 25.0 MB\\
CPU overhead & 0 B & 0 B & 0 B & 0 B & 0 B & 0 B\\
GPU overhead & 1.6 MB & 1.6 MB & 6.1 MB & 6.2 MB & 6.2 MB & 6.2 MB\\
CPU time & 211.0 $\mu$s & 217.0 $\mu$s & 925.6 $\mu$s & 915.5 $\mu$s & 797.4 $\mu$s & 3.0 ms\\
GPU time (10) & 1.4 ms & 1.5 ms & 5.7 ms & 5.9 ms & 5.3 ms & 12.9 ms\\
\multicolumn{6}{c}{\textbf{Skinning Decomposition (\csfb)}}\\
MXE &   0.719 cm &   0.754 cm &   0.632 cm &   0.428 cm &   0.535 cm &   0.227 cm\\
MAE & 0.00378 cm & 0.00277 cm & 0.00341 cm & 0.00295 cm & 0.00354 cm & 0.00147 cm\\
EAD & 0.00502 rad & 0.00511 rad & 0.00395 rad & 0.00349 rad & 0.00601 rad & 0.00365 rad\\
\multicolumn{6}{c}{\textbf{Skinning Decomposition (\csfb) + new loss}}\\
size & 302.3 KB & 302.3 KB & 1.0 MB & 1.0 MB & 1017.9 KB & 1.1 MB\\
CPU overhead & 9.4 KB & 9.4 KB & 9.4 KB & 9.4 KB & 9.4 KB & 9.4 KB\\
GPU overhead & 9.4 KB & 9.4 KB & 9.4 KB & 9.4 KB & 9.4 KB & 9.4 KB\\
CPU time & 282.1 $\mu$s & 282.0 $\mu$s & 1.1 ms & 1.1 ms & 943.5 $\mu$s & 1.1 ms\\
GPU time (10) & 248.1 $\mu$s & 249.3 $\mu$s & 493.7 $\mu$s & 493.6 $\mu$s & 510.2 $\mu$s & 570.6 $\mu$s\\
MXE &   0.408 cm &   0.325 cm &   0.499 cm &   0.377 cm &   0.588 cm &   0.187 cm\\
MAE & 0.00356 cm & 0.00262 cm & 0.00335 cm & 0.00276 cm & 0.00348 cm & 0.00139 cm\\
EAD & 0.00487 rad & 0.00488 rad & 0.00393 rad & 0.00344 rad & 0.00629 rad & 0.00365 rad\\
\multicolumn{6}{c}{\textbf{Sparse Factorization (\mf)}}\\
size & 177.5 KB & 153.3 KB & 561.1 KB & 561.2 KB & 655.1 KB & 707.6 KB\\
CPU overhead & 2.3 KB & 2.3 KB & 2.3 KB & 2.3 KB & 2.3 KB & 2.3 KB\\
GPU overhead & 43.1 KB & 42.2 KB & 93.3 KB & 89.6 KB & 109.8 KB & 110.7 KB\\
CPU time & 94.1 $\mu$s & 84.8 $\mu$s & 378.2 $\mu$s & 383.3 $\mu$s & 373.5 $\mu$s & 447.1 $\mu$s\\
GPU time (10) & 224.7 $\mu$s & 218.4 $\mu$s & 316.2 $\mu$s & 309.4 $\mu$s & 334.7 $\mu$s & 363.1 $\mu$s\\
MXE &   0.386 cm &   0.339 cm &   0.327 cm &   0.196 cm &   0.292 cm &   0.153 cm\\
MAE &  0.0033 cm &  0.0025 cm & 0.00243 cm & 0.00189 cm & 0.00297 cm & 0.00155 cm\\
EAD & 0.00484 rad & 0.00486 rad &  0.0039 rad & 0.00335 rad & 0.00444 rad & 0.00318 rad\\
\multicolumn{6}{c}{\textbf{Quantized Sparse Factorization  (\qmf) small}}\\
size & 75.0 KB & 65.4 KB & 274.6 KB & 274.7 KB & 316.6 KB & 343.1 KB\\
CPU time & 104.8 $\mu$s & 88.9 $\mu$s & 541.3 $\mu$s & 550.7 $\mu$s & 566.8 $\mu$s & 681.8 $\mu$s\\
GPU time (10) & 207.9 $\mu$s & 206.9 $\mu$s & 349.2 $\mu$s & 347.1 $\mu$s & 355.6 $\mu$s & 395.5 $\mu$s\\
MXE &   0.386 cm &   0.341 cm &   0.327 cm &   0.196 cm &   0.292 cm &   0.153 cm\\
MAE & 0.00333 cm & 0.00252 cm & 0.00243 cm & 0.00189 cm & 0.00297 cm & 0.00155 cm\\
EAD &  0.0053 rad & 0.00518 rad & 0.00398 rad & 0.00337 rad & 0.00453 rad & 0.00319 rad\\
\multicolumn{6}{c}{\textbf{Quantized Sparse Factorization  (\qmf) fast}}\\
size & 75.0 KB & 65.4 KB & 346.2 KB & 346.3 KB & 401.2 KB & 434.2 KB\\
CPU time & 104.8 $\mu$s & 88.9 $\mu$s & 415.8 $\mu$s & 423.0 $\mu$s & 421.9 $\mu$s & 500.9 $\mu$s\\
GPU time (10) & 207.9 $\mu$s & 206.9 $\mu$s & 294.2 $\mu$s & 292.9 $\mu$s & 305.8 $\mu$s & 332.5 $\mu$s\\
MXE &   0.386 cm &   0.341 cm &   0.327 cm &   0.196 cm &   0.292 cm &   0.153 cm\\
MAE & 0.00333 cm & 0.00252 cm & 0.00243 cm & 0.00189 cm & 0.00297 cm & 0.00155 cm\\
EAD &  0.0053 rad & 0.00518 rad &  0.0039 rad & 0.00335 rad & 0.00444 rad & 0.00318 rad\\

      \hline
      \end{tabular}
    \end{center}
    \caption{Results. Note that GPU time is for animating 10 faces concurrently.}
    \label{tab:FullResults}
    \end{table*}
    

%--
%
\begin{table}[t]
\begin{center}
\begin{tabular}{ l r r r r }
  \hline
  model & num Vertices & num BS & Dens BS & Sparse BS\\
  \hline
  aura & 5944 & 267 & 18.2 MB & 6.4 MB \\
  jupiter & 5944 & 319 & 21.7 MB & 6.4 MB \\
  boz & 23735 & 253 & 68.7 MB & 24.4 MB \\
  proteus & 23735 & 287 & 78.0 MB & 25.0 MB \\
  louise & 20053 & 253 & 58.1 MB & 24.7 MB \\
  carlos & 23735 & 287 & 78.0 MB & 25.0 MB \\

  \hline
  \end{tabular}
\end{center}
\caption{Test Models}
\label{tab:TestModels}
\end{table} 

\begin{figure}
    \includegraphics[width=0.31\linewidth]{aura_forehead_orig.png}
    \includegraphics[width=0.31\linewidth]{aura_forehead_SD0.png}
    \includegraphics[width=0.31\linewidth]{aura_forehead_SD1.png} \\
    \vspace{-1em}
    \includegraphics[width=0.31\linewidth, trim={0cm 39cm 17cm 12cm},clip]{aura_forehead_orig.png}
    \includegraphics[width=0.31\linewidth, trim={0cm 39cm 17cm 12cm},clip]{aura_forehead_SD0.png}
    \includegraphics[width=0.31\linewidth, trim={0cm 39cm 17cm 12cm},clip]{aura_forehead_SD1.png}
    \vspace{1em}
    \caption{\csfb results left to right: sparse blendshapes, old loss, new loss }
    \label{fig:CSFBAura}
\end{figure}


\begin{figure}
    \includegraphics[width=\linewidth]{compressionRatesDiagram.pdf}
    \caption{Compression Rates relative to sparse blendshapes size}
    \label{fig:Compression}
\end{figure}


%\begin{figure}
%    \includegraphics[width=\linewidth]{example-image-a}
%    \caption{Errors}
%    \label{Fig:Errors}
%\end{figure}

%%
    \begin{table*}[t]
    \begin{center}
    \begin{tabular}{ l  r r r r r r }
\hline
  & Aura & Jupiter & Bowen & Proteus & Louise & Carlos\\
\hline
num vertices & 5944 & 5944 & 23735 & 23735 & 20053 & 23735\\
num BS & 267 & 319 & 253 & 287 & 253 & 287\\
dense BS & 18.2 MB & 21.7 MB & 68.7 MB & 78.0 MB & 58.1 MB & 78.0 MB\\
\multicolumn{6}{c}{\textbf{Sparse Blendshapes}}\\
size & 6.4 MB & 6.4 MB & 24.4 MB & 25.0 MB & 24.7 MB & 25.0 MB\\
CPU overhead & 0 B & 0 B & 0 B & 0 B & 0 B & 0 B\\
GPU overhead & 1.6 MB & 1.6 MB & 6.1 MB & 6.2 MB & 6.2 MB & 6.2 MB\\
CPU time & 211.0 $\mu$s & 217.0 $\mu$s & 925.6 $\mu$s & 915.5 $\mu$s & 797.4 $\mu$s & 3.0 ms\\
GPU time (10) & 1.4 ms & 1.5 ms & 5.7 ms & 5.9 ms & 5.3 ms & 12.9 ms\\
\multicolumn{6}{c}{\textbf{Skinning Decomposition (\csfb)}}\\
MXE &   0.719 cm &   0.754 cm &   0.632 cm &   0.428 cm &   0.535 cm &   0.227 cm\\
MAE & 0.00378 cm & 0.00277 cm & 0.00341 cm & 0.00295 cm & 0.00354 cm & 0.00147 cm\\
EAD & 0.00502 rad & 0.00511 rad & 0.00395 rad & 0.00349 rad & 0.00601 rad & 0.00365 rad\\
\multicolumn{6}{c}{\textbf{Skinning Decomposition (\csfb) + new loss}}\\
size & 302.3 KB & 302.3 KB & 1.0 MB & 1.0 MB & 1017.9 KB & 1.1 MB\\
CPU overhead & 9.4 KB & 9.4 KB & 9.4 KB & 9.4 KB & 9.4 KB & 9.4 KB\\
GPU overhead & 9.4 KB & 9.4 KB & 9.4 KB & 9.4 KB & 9.4 KB & 9.4 KB\\
CPU time & 282.1 $\mu$s & 282.0 $\mu$s & 1.1 ms & 1.1 ms & 943.5 $\mu$s & 1.1 ms\\
GPU time (10) & 248.1 $\mu$s & 249.3 $\mu$s & 493.7 $\mu$s & 493.6 $\mu$s & 510.2 $\mu$s & 570.6 $\mu$s\\
MXE &   0.408 cm &   0.325 cm &   0.499 cm &   0.377 cm &   0.588 cm &   0.187 cm\\
MAE & 0.00356 cm & 0.00262 cm & 0.00335 cm & 0.00276 cm & 0.00348 cm & 0.00139 cm\\
EAD & 0.00487 rad & 0.00488 rad & 0.00393 rad & 0.00344 rad & 0.00629 rad & 0.00365 rad\\
\multicolumn{6}{c}{\textbf{Sparse Factorization (\mf)}}\\
size & 177.5 KB & 153.3 KB & 561.1 KB & 561.2 KB & 655.1 KB & 707.6 KB\\
CPU overhead & 2.3 KB & 2.3 KB & 2.3 KB & 2.3 KB & 2.3 KB & 2.3 KB\\
GPU overhead & 43.1 KB & 42.2 KB & 93.3 KB & 89.6 KB & 109.8 KB & 110.7 KB\\
CPU time & 94.1 $\mu$s & 84.8 $\mu$s & 378.2 $\mu$s & 383.3 $\mu$s & 373.5 $\mu$s & 447.1 $\mu$s\\
GPU time (10) & 224.7 $\mu$s & 218.4 $\mu$s & 316.2 $\mu$s & 309.4 $\mu$s & 334.7 $\mu$s & 363.1 $\mu$s\\
MXE &   0.386 cm &   0.339 cm &   0.327 cm &   0.196 cm &   0.292 cm &   0.153 cm\\
MAE &  0.0033 cm &  0.0025 cm & 0.00243 cm & 0.00189 cm & 0.00297 cm & 0.00155 cm\\
EAD & 0.00484 rad & 0.00486 rad &  0.0039 rad & 0.00335 rad & 0.00444 rad & 0.00318 rad\\
\multicolumn{6}{c}{\textbf{Quantized Sparse Factorization  (\qmf) small}}\\
size & 75.0 KB & 65.4 KB & 274.6 KB & 274.7 KB & 316.6 KB & 343.1 KB\\
CPU time & 104.8 $\mu$s & 88.9 $\mu$s & 541.3 $\mu$s & 550.7 $\mu$s & 566.8 $\mu$s & 681.8 $\mu$s\\
GPU time (10) & 207.9 $\mu$s & 206.9 $\mu$s & 349.2 $\mu$s & 347.1 $\mu$s & 355.6 $\mu$s & 395.5 $\mu$s\\
MXE &   0.386 cm &   0.341 cm &   0.327 cm &   0.196 cm &   0.292 cm &   0.153 cm\\
MAE & 0.00333 cm & 0.00252 cm & 0.00243 cm & 0.00189 cm & 0.00297 cm & 0.00155 cm\\
EAD &  0.0053 rad & 0.00518 rad & 0.00398 rad & 0.00337 rad & 0.00453 rad & 0.00319 rad\\
\multicolumn{6}{c}{\textbf{Quantized Sparse Factorization  (\qmf) fast}}\\
size & 75.0 KB & 65.4 KB & 346.2 KB & 346.3 KB & 401.2 KB & 434.2 KB\\
CPU time & 104.8 $\mu$s & 88.9 $\mu$s & 415.8 $\mu$s & 423.0 $\mu$s & 421.9 $\mu$s & 500.9 $\mu$s\\
GPU time (10) & 207.9 $\mu$s & 206.9 $\mu$s & 294.2 $\mu$s & 292.9 $\mu$s & 305.8 $\mu$s & 332.5 $\mu$s\\
MXE &   0.386 cm &   0.341 cm &   0.327 cm &   0.196 cm &   0.292 cm &   0.153 cm\\
MAE & 0.00333 cm & 0.00252 cm & 0.00243 cm & 0.00189 cm & 0.00297 cm & 0.00155 cm\\
EAD &  0.0053 rad & 0.00518 rad &  0.0039 rad & 0.00335 rad & 0.00444 rad & 0.00318 rad\\

      \hline
      \end{tabular}
    \end{center}
    \caption{Results. Note that GPU time is for animating 10 faces concurrently.}
    \label{tab:FullResults}
    \end{table*}
    
% Epic Games meta Human whitepaper https://cdn2.unrealengine.com/rig-logic-whitepaper-v2-5c9f23f7e210.pdf

\section{Conclusion \& Future Work}
We have presented a novel blendshape compression algorithm that offers a significant improvement in terms of storage efficiency and computational performance for facial animation. By leveraging sparse matrix factorization, a new loss function, and quantization techniques, we are able to achieve higher compression ratios without sacrificing the essential features of the original blendshapes. This work has important implications for the development of more efficient and effective facial animation systems, enabling the creation of more realistic and engaging virtual characters.  These features should provide benefit in almost any scenario requiring facial animation, but will be especially useful for low-power and low-memory devices like those used for augmented reality.

One trade-off with our technique is that we need new run-time kernels, as standard linear blend skinning routines cannot be used.  However the algorithm is quite simple, and requires only small modifications and a small amount of additional data to enable the algorithm to become amenable to parallel processing (\Algorithm{alg:CudaCode}).

Additionally, the outcome of optimization is somewhat dependent on initial random seeds (\csfb shares this trait).  We have found, however, that once on the optimization path, convergence trajectory is somewhat predictable (see \Figure{fig:InitErrors}), which allows us to run several optimization steps with different random seeds, choose the best one, and continue optimizing with only that seed.  In practice this makes the process highly reliable while keeping optimization time reasonable.  It may be that future work can find a better way to select/cull random values in the matrices, leading to even more robust results with less computation.

One limitation of our work is that we have three hyperparameters that require per-model tuning: $\alpha$, how influential the Laplacian term is in our loss function, $\gamma$, how strong the influence of the wrinkles map is, and $\numNonzeroBC$, the number of non-zero values in the $\matB$ and $\matC$ matrices.  Future work could explore ways to automatically derive these based on desired compression vs quality.

Finally, while our wrinkles map technique does a very good job of driving non-zero values to visually important portions of the matrices, it would be interesting to investigate whether a more principled approach might be possible.

\begin{acks}

\end{acks}
% \clearpage
\bibliographystyle{ACM-Reference-Format}

\vspace{-1.5mm}
\bibliography{reference}
% \clearpage

\begin{figure*}[htbp]
  \subfloat[Aura]{\includegraphics[width=0.29\linewidth]{modelAura.png}}\hfill
  \subfloat[Bowen]{\includegraphics[width=0.3\linewidth]{modelBoz.png}}\hfill
 \subfloat[Jupiter]{\includegraphics[width=0.31\linewidth]{modelJupiter.png}} \\
   \vspace{1em}
  \subfloat[Proteus]{\includegraphics[width=0.3\linewidth]{modelProteus.png}}\hfill
  \subfloat[Louise]{\includegraphics[width=0.3\linewidth]{modelLouise.png}}\hfill
  \subfloat[Carlos]{\includegraphics[width=0.3\linewidth]{modelCarlos.png}}\\
    \vspace{1em}
\caption{Images rendered from the datasets used for experiments.}
\label{fig:Datasets}
\end{figure*}

\end{document}
