\section{Introduction}

% Blurb on why it's important to compress blendshapes
Users are consuming 3D graphics experiences on ever-shrinking devices in order to maximize user satisfaction; the ultimate goal is an eyeglasses form factor with minimal weight and volume. However, weight and power limitations require us to rethink classical computer graphics techniques that have served us well in the previous era of personal computers and game consoles. In this paper we focus on blendshape-based facial animation and present a new blendshape compression algorithm which promises to produce facial models suitable for upcoming computing platforms.

% Most popular is to approximate blendshapes via a skinning decomposition, but that's not ideal: skinning was designing for large articulated deformations
The problem of blendshape compression is not new; high visual fidelity demands hundreds or even thousands of blendshapes. Since classical quantization or sparsification methods provide only moderate data reduction \cite{seo2011compression}, many recent systems instead approximate blendshapes via linear blend skinning decomposition \cite{james2005skinning, dembones}. However, skinning is primarily intended for full-body deformations with large joint rotations (e.g.,  arms), and is not ideal for subtle facial deformations handled via numerically small but visually important correctives \cite{lewis2014practice}. Attempts to accurately fit the subtle facial deformations with skinning typically require a relatively large number of non-zero skinning weights per vertex (8+), which can be problematic on resource-constrained platforms \cite{kavan2024compressed}.

In this paper we argue for a different, lower-level approach, based on quantized sparse matrix factorization. %If the input matrix of blendshape deltas is denoted as $\matA\in\reals^{3 \numBS \times \numVertices}$, our goal is to approximate this matrix via a matrix product of $\matB\in\reals^{3\numBS \times \numColumnB} $ and $\matC\in\reals^{\numColumnB \times \numVertices}$, where $\matB$ and $\matC$ are sparse and $\numColumnB$ is a hyper-parameter. 
This factorization is related to linear blend skinning (which can also be formulated as a special type of matrix factorization \cite{kavan2010fast}), but distinct in the following ways: we decidedly forgo the structure of 3D skinning transformations and skinning weights (with the associated partition-of-unity constraints and per-vertex-nonzeros limits). 
%the nonzero values can be distributed arbitrarily among $\matB$ and $\matC$ \LK{is this actually true or do we impose some per-row / per-column limits?} \RF{as part of wrinkles correction we introduce upper limit for number of non-zero values per column of matrix C (per vertex). If we don't use wrinkle corrections we don ton impose any per column limits}. 
Sparse factorization is a more elementary abstraction than skinning and, in practice, allows higher flexibility in capturing high-frequency details in the input shapes $\matA$. Specifically, our results show well over a 3$\times$ reduction of storage compared to skinning decomposition-based compression with equal or even slightly improved approximation accuracy 

Given the matrix $\matA$, we solve for sparse matrices $\matB$ and $\matC$, such that $\matB\matC = \matA$, using a projected Adam optimizer \cite{kingma2014adam}, where the projection corresponds to our desired sparsification. To avoid non-smooth, visually disturbing errors $\matA - \matB \matC$, previous skinning decomposition work usually adds Laplacian regularizers \cite{le2014robust,kavan2024compressed}. This works, but attenuates the high-frequency details of $\matA$ such as wrinkles. Instead, we propose a new loss function, and a new function to bias toward retaining data in visually important regions, leading to more accurate approximation of details such as wrinkles (\Figure{fig:Teaser}) 

Another important opportunity to further reduce memory presents itself in the quantization of non-zero coefficients of $\matB$ and $\matC$. We demonstrate that by quantizing the non-zeros of $\matB$ and $\matC$ to 8- or 10.66-bits (11, 11, and 10 bits packed into a 32-bit integer), we introduce only negligible error, but reduce memory even more to achieve an overall 3$\times$ to 5$\times$ storage savings over a state-of-the-art skinning decomposition method with sparse skinning transformations \cite{kavan2024compressed} while simultaneously also \textit{improving} the approximation accuracy. For these reasons we argue that quantized sparse matrix factorization is an important technique for delivering high quality blendshape animation on resource-constrained platforms.

Our key contributions are that we:
\begin{itemize}
    \item depart from skinning decomposition approaches and introduce a new way to compress blendshapes via factorization into sparse matrices;
    \item propose a new loss function and an additional technique that combine to preserve high-frequency details like wrinkles;
    \item and demonstrate quantization of the non-zero matrix values further to reduce memory while preserving sufficient precision.
\end{itemize}
